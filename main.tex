%-------------------------
% Resume in Latex
% Author : Jake Gutierrez
% Based off of: https://github.com/sb2nov/resume
% License : MIT
%------------------------

\documentclass[letterpaper,11pt]{article}

\usepackage{latexsym}
\usepackage[empty]{fullpage}
\usepackage{titlesec}
\usepackage{marvosym}
\usepackage[usenames,dvipsnames]{color}
\usepackage{verbatim}
\usepackage{enumitem}
\usepackage[hidelinks]{hyperref}
\usepackage{fancyhdr}
\usepackage[english]{babel}
\usepackage{tabularx}
\usepackage{fontawesome5}
\usepackage{multicol}
\usepackage{hyperref}
\usepackage{xcolor}

\usepackage[T1]{fontenc}

\setlength{\multicolsep}{-3.0pt}
\setlength{\columnsep}{-1pt}
\input{glyphtounicode}
\setlength{\footskip}{20pt}


%----------FONT OPTIONS----------
% sans-serif
% \usepackage[sfdefault]{FiraSans}
% \usepackage[sfdefault]{roboto}
% \usepackage[sfdefault]{noto-sans}
% \usepackage[default]{sourcesanspro}

% serif
% \usepackage{CormorantGaramond}
% \usepackage{charter}


\pagestyle{fancy}
\fancyhf{} % clear all header and footer fields
\fancyfoot{}
\renewcommand{\headrulewidth}{0pt}
\renewcommand{\footrulewidth}{0pt}

% Adjust margins
\addtolength{\oddsidemargin}{-0.6in}
\addtolength{\evensidemargin}{-0.5in}
\addtolength{\textwidth}{1.19in}
\addtolength{\topmargin}{-.7in}
\addtolength{\textheight}{1.4in}

\urlstyle{same}

\raggedbottom
\raggedright
\setlength{\tabcolsep}{0in}

% Sections formatting
\titleformat{\section}{
  \vspace{-4pt}\scshape\raggedright\large\bfseries
}{}{0em}{}[\color{black}\titlerule \vspace{-5pt}]

% Ensure that generate pdf is machine readable/ATS parsable
\pdfgentounicode=1

%-------------------------
% Custom commands
\newcommand{\resumeItem}[1]{
  \item\small{
    {#1 \vspace{-2pt}}
  }
}

\newcommand{\pubItem}[1]{
  \item\small{
    {#1 \vspace{-2pt}}
  }
}

\newcommand{\classesList}[4]{
    \item\small{
        {#1 #2 #3 #4 \vspace{-2pt}}
  }
}

% \newcommand{\resumeSubheading}[4]{
%   \vspace{-2pt}\item
%     \begin{tabular*}{1.0\textwidth}[t]{l@{\extracolsep{\fill}}r}
%       \textbf{#1} & \textbf{\small #2} \\
%       \textit{\small#3} & \textit{\small #4} \\
%      % \textbf{\small#5} & \textbf{\small #6} \\

%     \end{tabular*}\vspace{-10pt}
% }

\newcommand{\resumeSubheading}[4]{
  \vspace{-2pt}\item
    \begin{tabular*}{1.0\textwidth}[t]{l@{\extracolsep{\fill}}r}
      \textbf{#1} & \textbf{\small #2} \\
      \textit{\small#3} & \makebox[0pt][r]{\textit{\small #4}} \\
     % \textbf{\small#5} & \textbf{\small #6} \\

    \end{tabular*}\vspace{-10pt}
}
% \newcommand{\experience}[6]{
%   \vspace{-2pt}\item
%     \begin{tabular*}{1.0\textwidth}[t]{l@{\extracolsep{\fill}}r}
%       \textbf{#1} & \textbf{\small #2} \\
%       \textit{\small#3} & \textit{\small #4} & \textbf{\small#5} \\

%     \end{tabular*}\vspace{-10pt}
% }
\newcommand{\honor}[2]{
  \vspace{-2pt}\item
    \begin{tabular*}{1.0\textwidth}[t]{l@{\extracolsep{\fill}}r}
      \small #1 & #2 \\
    \end{tabular*}\vspace{-15pt}
}

\newcommand{\honorLast}[2]{
  \vspace{-2pt}\item
    \begin{tabular*}{1.0\textwidth}[t]{l@{\extracolsep{\fill}}r}
      \small #1 & #2 \\
    \end{tabular*}\vspace{-5pt}
}


\newcommand{\resumeEdu}[5]{
  \vspace{-2pt}\item
    \begin{tabular*}{1.0\textwidth}[t]{l@{\extracolsep{\fill}}r}
      \textbf{#1} & \textbf{\small #2} \\
      \textit{\small#3} & \textit{\small #4} \\
      \textit{\small#5}\\

    \end{tabular*}\vspace{-10pt}
}


\newcommand{\resumeSubSubheading}[2]{
    \item
    \begin{tabular*}{0.97\textwidth}{l@{\extracolsep{\fill}}r}
      \textit{\small#1} & \textit{\small #2} \\
    \end{tabular*}\vspace{-7pt}
}

\newcommand{\resumeProjectHeading}[2]{
    \item
    \begin{tabular*}{1.001\textwidth}{l@{\extracolsep{\fill}}r}
      \small#1 & \textbf{\small #2}\\
    \end{tabular*}\vspace{-7pt}
}

\newcommand{\resumeSubItem}[1]{\resumeItem{#1}\vspace{-2pt}}

\renewcommand\labelitemi{$\vcenter{\hbox{\tiny$\bullet$}}$}
\renewcommand\labelitemii{$\vcenter{\hbox{\tiny$\bullet$}}$}

\newcommand{\resumeSubHeadingListStart}{\begin{itemize}[leftmargin=0.0in, label={}]}
\newcommand{\resumeSubHeadingListEnd}{\end{itemize}}
\newcommand{\resumeItemListStart}{\begin{itemize}}
\newcommand{\resumeItemListEnd}{\end{itemize}\vspace{-5pt}}

%-------------------------------------------
%%%%%%  RESUME STARTS HERE  %%%%%%%%%%%%%%%%%%%%%%%%%%%%


\begin{document}

%----------HEADING----------
% \begin{tabular*}{\textwidth}{l@{\extracolsep{\fill}}r}
%   \textbf{\href{http://sourabhbajaj.com/}{\Large Sourabh Bajaj}} & Email : \href{mailto:sourabh@sourabhbajaj.com}{sourabh@sourabhbajaj.com}\\
%   \href{http://sourabhbajaj.com/}{http://www.sourabhbajaj.com} & Mobile : +1-123-456-7890 \\
% \end{tabular*}

\begin{center}
    {\Huge \scshape Hanrong Zhang} \\ \vspace{1pt}
    % \small Address: No.718, Haizhou East Road, Haining City, Zhejiang Province, China, 314400 \\ \vspace{1pt}
    \small Tel: (+1)312-479-7822 $|$
    \small  {Email: zhanghr0709@gmail.com}
    $|$ \small {WeChat: henry\_zhang0709}
    % \href{linkedin.com/in/hanrong-zhang-08a693193}{Linkedin:\ \underline{linkedin.com/in/hanrong-zhang-08a693193}}  ~
    % \href{https://github.com/}{Github:\ \underline{github.com/username}}
    $|$ \small  \href{https://zhang-henry.github.io}{\textcolor{blue}{\underline{Homepage}}}
    $|$ \small \href{https://scholar.google.com/citations?user=qG5_O40AAAAJ&hl=zh-CN}{\textcolor{blue}{\underline{Google Scholar}}}

    % \vspace{-8pt}
\end{center}


%-----------EDUCATION-----------

% \section{Research Interests}
% \begin{enumerate}[itemsep=1pt, topsep=1pt, parsep=1pt, leftmargin=*, label={[\arabic*]}]
%     \item Trustworthy Machine Learning: LLM (Agent) Safety and Security, Adversarial Attack and Defense;
%     \item Machine Learning: Deep Learning, e.g., Incremental Learning, Out-of-Distribution Detection; Knowledge Graph
%     \item AI/ML+X: AI for Industrial Fault Diagnosis, AI for Healthcare
% \end{enumerate}



\section{Education}
  \resumeSubHeadingListStart
    \resumeEdu
      {University of Illinois Chicago}{Aug. 2025 -- Present}
      {Ph.D. Student in Computer Science}{Chicago, USA}
        {Big Data and Social Computing Lab; Supervisor: \href{https://cs.uic.edu/profiles/philip-yu/}{\textcolor{blue}{Prof. Philip S. Yu}}}
    \resumeEdu
      {Zhejiang University}{Sep. 2022 -- Mar. 2025}
      {MEng. Computer Engineering}{Hangzhou, China}
        {Ranking: 1/82; GPA: 93/100; Supervisor: \href{https://person.zju.edu.cn/en/hwang}{\textcolor{blue}{Prof. Hongwei Wang}}}
    \resumeSubheading
      {University of Leeds}{Sep. 2018 -- Jun. 2022}
      {BSc. Computer Science - First Class Honors Degree}{Leeds, United Kingdom}
    \resumeEdu
      {Southwest Jiaotong University}{Sep. 2018 -- Jun. 2022}
      {BEng. Computer Science and Technology}{Chengdu, China}
      {Ranking: 1/74; GPA: 3.81/4.0, 92/100;
      Supervisor: \href{https://scholar.google.com/citations?user=CQ1HneMAAAAJ}{\textcolor{blue}{Prof. Tianrui Li}}
      }
  \resumeSubHeadingListEnd
% \vspace{0.1pt}

%------RELEVANT COURSEWORK-------
% \section{Relevant Coursework}
%     %\resumeSubHeadingListStart
%         \begin{multicols}{4}
%             \begin{itemize}[itemsep=-5pt, parsep=3pt]
%                 \item\small Data Structures
%                 \item Software Methodology
%                 \item Algorithms Analysis
%                 \item Database Management
%                 \item Artificial Intelligence
%                 \item Internet Technology
%                 \item Systems Programming
%                 \item Computer Architecture
%             \end{itemize}
%         \end{multicols}
%         \vspace*{2.0\multicolsep}
%     %\resumeSubHeadingListEnd

\section{Selected Papers(*Equal Contribution)}

\noindent\textbf{Trustworthy Machine Learning:}
\begin{enumerate}[itemsep=1pt, topsep=1pt, parsep=1pt, leftmargin=*, label={[\arabic*]}]
    \item \small{\textbf{Hanrong Zhang}, Jingyuan Huang, Kai Mei, Yifei Yao, Zhenting Wang, Chenlu Zhan, Hongwei Wang, Yongfeng Zhang, \textit{Agent Security Bench (ASB): Formalizing and Benchmarking Attacks and Defenses in LLM-based Agents}, \textbf{ICLR 2025}, Singapore, Apr. 2025. \href{https://arxiv.org/abs/2410.02644}{\textcolor{blue}{[Paper]}} \href{https://github.com/agiresearch/ASB}{\textcolor{blue}{[Code]}} \href{https://luckfort.github.io/ASBench/}{\textcolor{blue}{[Website]}}
    % (\textit{Working with Prof. \href{https://www.yongfeng.me/}{Yongfeng Zhang}, Rutgers})
    }
    \item \small{\textbf{Hanrong Zhang}, Zhenting Wang, Tingxu Han, Mingyu Jin, Chenlu Zhan, Mengnan Du, Hongwei Wang, Shiqing Ma, \textit{Invisible Backdoor Attack in Self-supervised Learning}, \textbf{CVPR 2025}, Nashville, USA, Jun. 2025. \href{https://arxiv.org/abs/2405.14672}{\textcolor{blue}{[Paper]}} \href{https://github.com/Zhang-Henry/INACTIVE}{\textcolor{blue}{[Code]}}
    % (\textit{Working with Prof. \href{https://people.cs.umass.edu/~shiqingma/}{Shiqing Ma}, UMass Amherst; Prof. \href{https://mengnandu.com/}{Mengnan Du}, NJIT})
    }
\end{enumerate}

\noindent\textbf{Machine Learning and Data Mining:}
\begin{enumerate}[itemsep=1pt, topsep=1pt, parsep=1pt, leftmargin=*, label={[\arabic*]}]
    \item \small{\textbf{Hanrong Zhang}, Yifei Yao, Zixuan Wang, Jiayuan Su, Mengxuan Li, Peng Peng, Hongwei Wang, \textit{Class Incremental Fault Diagnosis under Limited Fault Data via Supervised Contrastive Knowledge Distillation}, \textbf{IEEE Transactions on Industrial Informatics}. (IF=12.3, JCR Q1 SCI) \href{https://arxiv.org/pdf/2501.09525}{\textcolor{blue}{[Paper]}}}
    \item \small{Xingyue Wang*, \textbf{Hanrong Zhang}*, Xinlong Qiao, Ke Ma, Shuting Tao, Peng Peng, Hongwei Wang, \textit{Generalized Out-of-distribution Fault Diagnosis (GOOFD) via Internal Contrastive Learning}, \textbf{IEEE Transactions on Industrial Informatics}. (IF=12.3, JCR Q1 SCI) \href{https://ieeexplore.ieee.org/abstract/document/10510599}{\textcolor{blue}{[Paper]}}}
    \item \small{Peng Peng*, \textbf{Hanrong Zhang}*, Xinyue Wang, Wanqiu Huang, Hongwei Wang, \textit{Imbalanced Chemical Process Fault Diagnosis Using Balancing GAN With Active Sample Selection}, \textbf{IEEE Sensors Journal}. (IF=4.3, JCR Q1 SCI) \href{https://ieeexplore.ieee.org/abstract/document/10114639}{\textcolor{blue}{[Paper]}}}
    \item \small{Wanqiu Huang, \textbf{Hanrong Zhang}, Peng Peng, Hongwei Wang, \textit{Multi-gate Mixture-of-Expert Combined with Synthetic Minority Over-sampling Technique for Multimode Imbalanced Fault Diagnosis}, \textbf{IEEE International Conference on Computer Supported Cooperative Work in Design 2023}. \textbf{(Best Paper Award Finalist)} \href{https://ieeexplore.ieee.org/abstract/document/10152774}{\textcolor{blue}{[Paper]}}}
    \item \small{\textbf{Hanrong Zhang}, Xinyue Wang, Jiabao Pan, Hongwei Wang, \textit{SAKA: an intelligent platform for semi-automated knowledge graph construction and application}, \textbf{Service Oriented Computing and Applications}. \href{https://link.springer.com/article/10.1007/s11761-023-00371-x}{\textcolor{blue}{[Paper]}}}
    \item \small{\textbf{Hanrong Zhang}, Xinyue Wang, Bo Qin, Hongwei Wang, \textit{An Intelligent System for Semantic Information Extraction and Knowledge Graph Construction from Multi-Type Data Sources}, \textbf{IEEE ICEBE 2022}. \href{https://ieeexplore.ieee.org/abstract/document/10035077}{\textcolor{blue}{[Paper]}}}
    % \item \small{\textbf{H. Zhang}, Y. Yao, et al., From Uncertainty to Clarity: Uncertainty-Guided Class-Incremental Learning for Limited Biomedical Samples via Semantic Expansion, arXiv preprint.} (Under Review)
\end{enumerate}

\vspace{5pt}
\small{\textit{Full publication list is available on \href{https://scholar.google.com/citations?user=qG5_O40AAAAJ&hl=zh-CN}{\textcolor{blue}{Google Scholar}}.}}

\section{Internship}
  \resumeSubHeadingListStart

    \resumeSubheading
      {RL for Human-Agent Multi-turn Interaction}{Alibaba Group}
      {Research Intern}{Hangzhou, May 2025 -- Aug. 2025}
      \resumeItemListStart
        \resumeItem{To address the challenges of sparse rewards, limited data, and environment instability in LLM agent-human multi-turn interactions, I develop scalable simulation environments with various tool-call scenarios (e.g., retail, airline) to facilitate multi-turn interactions.}
        \resumeItem{I propose a novel Tool Dependency Graph structure to model tool dependencies, enabling high-quality, multi-turn tool-call dialogues and generating ground truth trajectories, which is used for verifiable reward signal computation in multi-turn conversations.}
        \resumeItem{I leverage this data to conduct multi-turn GRPO training, resulting in significant improvements in the agent’s performance in tool usage during human-agent multi-turn interactions.}
      \resumeItemListEnd
  \resumeSubHeadingListEnd

% \vspace{-21pt}
\section{Selected Honors}
  \resumeSubHeadingListStart
    \honor {Outstanding Graduate in Zhejiang Province \& Zhejiang University}{Zhejiang University, 2025}
    \honor {National Scholarship for Graduate Students (Top 0.2\%)}{Zhejiang University, 2023 - 2024}
    \honor {National Scholarship for Graduate Students (Top 0.2\%)}{Zhejiang University, 2022 - 2023}
    \honor {National Scholarship for Undergraduate Students (Top 0.2\%)}{Southwest Jiaotong University, 2019 - 2020}
    \honor {Outstanding Graduate in Sichuan Province}{Southwest Jiaotong University, 2022}
    \honor {Pacemaker to Merit Student}{Southwest Jiaotong University, 2018 - 2021}
    \honor {Best Student in Computer Science (1/75)}{University of Leeds, 2020 - 2021}
    \honor {First-class full-ride Scholarship (1/75)}{University of Leeds, 2020 - 2021}
    \honorLast {Best Student Overall (1/300, 4 majors)}{University of Leeds, 2018 - 2019}
  \resumeSubHeadingListEnd

\section{Selected Competiton Awards}
  \resumeSubHeadingListStart
    \honor {Mathematical Modeling Contest for College Students $|$ \textit{National Second Prize} (Top 0.5\% of 45,000 teams)}{Sep. 2020}
    \honor {Students Service Outsourcing Innovation and Entrepreneurship Competition $|$ \textit{National Second Prize} (Top 3\%)}{Aug. 2020}
    \honor {MathorCup College Mathematical Modeling Competition $|$ \textit{First Prize} (Top 3\%)}{May 2020}
    \honor {May Day Mathematical Modeling Competition $|$ \textit{First Prize} (Top 3\%)}{May 2020}
    \honor {Asia-Pacific Mathematical Modeling Contest $|$ \textit{First Prize} (Top 3\%)}{Nov. 2019}
    \honor {Mathematical Modeling Competition in Southwest Jiaotong University $|$ \textit{First Prize} (Top 3\%)}{Nov. 2019}
    \honorLast {American College Student Mathematical Modeling Contest $|$ \textit{Honorable Mention} (Top 10\%)}{Jan. 2020}
  \resumeSubHeadingListEnd


\section{Research Experience}
  \resumeSubHeadingListStart

    \resumeSubheading
      {AgentSecurityBench (ASB): Formalizing and Benchmarking Attacks and Defenses in LLM-based Agents}{}
      {Research Intern@WISE Lab of Prof. \href{https://www.yongfeng.me/}{\textcolor{blue}{Yongfeng Zhang}}. \textcolor{blue}{\textbf{ICLR 2025 First Author}}}{Rutgers University}
      \resumeItemListStart

        \resumeItem{Introduce Agent Security Bench (ASB), a comprehensive framework designed to formalize, benchmark, and evaluate the attacks and defenses of LLM agents, including 10 scenarios (e.g., e-commerce, autonomous driving, finance), 10 agents targeting the scenarios, over 400 tools, 23 different types of attack/defense methods, and 7 evaluation metrics.}
        \resumeItem{Benchmark 10 prompt injection attacks, a memory poisoning attack, a novel Plan-of-Thought backdoor attack, and 11 corresponding defenses across 13 LLM backends with over 90,000 testing cases in total.}
        \resumeItem{Reveal critical vulnerabilities in different stages of agent operation, including system prompt, user prompt handling, tool usage, and memory retrieval, with the highest average attack success rate of 84.30\%, but limited effectiveness shown in current defenses.}
      \resumeItemListEnd

    \resumeSubheading
      {Towards Imperceptible Backdoor Attack in Self-supervised Learning}{}
      {Research Intern@Lab of Prof. \href{https://people.cs.umass.edu/~shiqingma/}{\textcolor{blue}{Shiqing Ma}}. \textcolor{blue}{\textbf{CVPR 2025 First Author}}}{UMass Amherst}
      \resumeItemListStart
        \resumeItem{Observe that existing imperceptible triggers designed for supervised classifiers have limited effectiveness in SSL, and current backdoor attacks on SSL, like BadEncoder, achieve high ASRs but rely on visible triggers.}
        \resumeItem{Find that the reason behind such ineffectiveness is the coupling feature-space distributions for the backdoor samples and augmented samples in the SSL models.}
        \resumeItem{Propose an imperceptible and effective backdoor attack in SSL by disentangling the distribution of backdoor samples and augmented samples in SSL, while constraining the stealthiness of the triggers during the optimization process.}
        \resumeItem{Extensive experiments on five datasets and six SSL algorithms with different augmentation ways demonstrate our attack is effective and stealthy, and can also be resilient to current SOTA backdoor defense methods.}
        % \resumeItem{This paper has been accepted by CVPR 2025.}

      \resumeItemListEnd

    \resumeSubheading
      {Few-shot Class Incremental Learning}{}
      {M.S. student. \textcolor{blue}{\textbf{IEEE Trans. on Industrial Informatics First Author}}}{Zhejiang University}
      \resumeItemListStart
        \resumeItem{Introduce a novel framework for class-incremental fault diagnosis under limited fault data. It addresses key challenges such as class imbalance, long-tailed distributions, and catastrophic forgetting.}
        \resumeItem{Propose supervised contrastive knowledge distillation to improve feature extraction from limited fault data while minimizing the forgetting of previously learned fault classes as new ones are introduced.}
        \resumeItem{Propose marginal exemplar selection, which prioritizes hard-to-classify edge-case samples for storage, helping the model recognize class boundaries and improve generalization.}
        % \resumeItem{This paper has been accepted by IEEE Transactions on Industrial Informatics (IF: 12.3).}
      \resumeItemListEnd

    % \resumeSubheading
    %   {Generalized Out-of-Distribution Fault Diagnosis via Internal Contrastive Learning}{}
    %   {M.S. student@DSKE Lab of Prof. \href{https://person.zju.edu.cn/en/hwang}{Hongwei Wang}. \textbf{Co-First Author}}{Zhejiang University, China, June 2022 -- Dec. 2022}
    %   \resumeItemListStart
    %     \resumeItem{We are the first to propose an integrated diagnostic system called GOOFD, including process monitoring, fault classification, and open set fault diagnosis tasks, which offers significant room for expansion and exploration.}
    %     \resumeItem{The paper introduces a novel integrated diagnosis method ICL-OD to 1) address the multi-tasks issue in the GOOFD framework and 2) learn more distinctive features for unknown classes based on internal contrastive learning and the Mahalanobis-distance approach.}
    %     \resumeItem{This paper has been accepted by IEEE Transactions on Industrial Informatics (IF: 12.3).}
    %   \resumeItemListEnd
  \resumeSubHeadingListEnd


\section{Academic Service}
  \resumeSubHeadingListStart
    \honor {Conference Reviewer: ICLR, CVPR}{2025 - Present}
    \honorLast {Journal Reviewer: IEEE Trans. on Reliability, Pattern Recognition}{2025 - Present}
  \resumeSubHeadingListEnd


% \section{Leadership / Extracurricular}
%   \resumeSubHeadingListStart
%     \honor {Teaching Assistant for UIUC ECE-448 Artificial Intelligence, with Prof. \href{https://scholar.google.com/citations?user=18O0OAwAAAAJ}{\textcolor{blue}{Mark Hasegawa-Johnson}}}{ZJU, 2022-2024}
%     \honor {Chief of Academic Technology and Innovation Base}{Southwest Jiaotong University, 2020-2021}
%     \honorLast {Peer Mentor \& Class Monitor}{Southwest Jiaotong University, 2018-2020}
%   \resumeSubHeadingListEnd

\vspace{10pt}
\begin{center}
\small{\textit{Last updated: \today}}
\end{center}

\end{document}
